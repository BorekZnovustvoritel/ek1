\clearpage
\section{Zásahy státu do cen (spotřební daň, cenový strop, subvence k ceně).}

\subsection{Spotřební daň}
\begin{itemize}
    \item daňová politika jako hlavní forma získávání rozpočtových příjmů zajišťuje zdroje 
    pro činnost státu, ale mění i chování jednotlivých subjektů na trhu
    \item rozpočtové výdaje - v součinnosti s~daněmi - také mění rozdělení příjmů ve společnosti 
    proti situaci, jak ji vytvořil trh a ovlivňují zde proces vytváření rovnováhy
    \item spotřební daň - vybírána z jednotky zboží, např. litru benzínu nebo z~krabičky cigaret
    \item kdo platí tuto daň – spotřebitel nebo výrobce? Odpověď v příkladu (podle příkladu oba, 
    výrobce trochu zlevní, aby se pořád výrobek kupoval a spotřebitel to pořád kupuje trochu dražší)
    \item rozdělení břemena mezi výrobce a spotřebitele závisí na relativní strmosti křivek poptávky a nabídky
    \begin{itemize}
        \item je-li poptávka hodně strmá a nabídka málo strmá, dopadá břemeno daně více na spotřebitele a méně na výrobce
        \item je-li poptávka málo strmá a nabídka hodně strmá, dopadá břemeno daně více na výrobce než spotřebitele
    \end{itemize}
\end{itemize}

\subsection{Cenový strop}
\begin{itemize}
    \item zásah státu do cenového systému, typ cenové regulace
    \item prodávající nesmí požadovat cenu vyšší, než je státem stanovený cenový strop
    \item politikové chtějí pomoci spotřebitelům, chránit je před vysokými cenami některých
    statků – zejména takových, které uspokojují základní potřeby
    \item monopolní trh - cenové stropy „odčerpají“ výrobcům jejich zisky a „věnují“ je spotřebitelům v podobě nižší ceny
    \item dokonalý trh - firmy dosahují dlouhodobě nulového ekonomického zisku, a pokud cenový strop sníží
    jejich příjmy pod náklady, jsou potom firmy nuceny omezit nabídku.
\end{itemize}

\subsection{Subvence k ceně (cenová podlaha)}
\begin{itemize}
    \item cenových podlaha - stanovování minimálních cen, za nichž bude statek prodáván spotřebitelům
    \item subvence - nástroj k udržení minimální ceny, stát doplácí výrobcům (producentům) k ceně statku
    určitou částku – subvenci
    \item subvence je vlastně záporná spotřební daň, má proto také opačné účinky než (kladná) spotřební daň: 
    na úkor daňových poplatníků zvyšuje cenu výrobci, snižuje cenu spotřebiteli a zvyšuje vyráběné
    a spotřebované množství statku
    \item prospěch ze subvencí mívají ti, kdo tvoří malou skupinu, ti totiž na daních zaplatí menší podíl 
    subvencí, než jaký z nich mají prospěch. To jsou většinou výrobci, kterých je méně než spotřebitelů.
    \item subvence k ceně vyvolávají neefektivnost
\end{itemize}