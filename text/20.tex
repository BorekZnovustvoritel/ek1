\clearpage
\section{Mikroekonomická politika státu a vládní selhání.}
\begin{itemize}
    \item Vliv státu na ekonomiku
    \item Nástroje hospodářské politiky:
    \begin{itemize}
        \item clo
        \item daně
        \item kvóty
        \item cenové regulace
        \item státní vlastnictví
        \item státní regulace (v sektoru veřejně prospěšných služeb, v dopravě, na finančních trzích,
        v zahraničním obchodu)
        \item antimonopolní politika (zakazuje určité druhy konkurenčního chování)
    \end{itemize}
    \item Řeší problémy jako inflace, nezaměstnanost
    \item \textbf{Podněty, které vedou orgány k zásahu:}
    \begin{itemize}
        \item neefektivní alokace výrobních faktorů a finální produkce
        \item přerozdělování příjmů
        \item konflikt mezi efektivitou a hodnotovým systémem společnosti
    \end{itemize}
\end{itemize}

\subsection{Neefektivní alokace výrobních faktorů a finální produkce}
\begin{itemize}
    \item Stát chce napomoct k efektivnější alokaci výrobních faktorů a finální produkce.
    \item Různé příčiny (nedokonalá konkurence, externality, nedokonalá informovanost)
    \item Mikroekonomický efekt - protimonopolní regulace
    \item Makroekonomický efekt - státní strukturální politika
\end{itemize}

\subsection{Přerozdělování příjmů}
\begin{itemize}
    \item Pokud stát nevnímá rozdělení příjmů trhem jako sociálně přijatelné
    \item Mikroekonomický efekt - vliv na dílčí trhy
    \item Makroekonomický efekt - vzájemné postavení ekonomických sektorů
\end{itemize}

\subsection{Konflikt mezi efektivitou a hodnotovým systémem společnosti}
\begin{itemize}
    \item Efektivní alokace může být v konfliktu s ostatními cíli společnosti
    \item Ochrana hodnotového  systému společnosti
    \item Např. prohibice
\end{itemize}

\subsection{Vládní selhání}
\begin{itemize}
    \item Činnost vlády negativně ovlivní rozdělení důchodů a nezlepší efektivnost
    \item Oblasti tržního selhání:
    \begin{itemize}
        \item čas (časové zpoždění)
        \item sledování vlastních zájmů
        \item vztah politiků k ekonomické teorii
        \item nevyužití politického kapitálu
    \end{itemize}
    \item Vláda chce maximalizovat zisk hlasů, ne ekonomickou prosperitu státu
\end{itemize}