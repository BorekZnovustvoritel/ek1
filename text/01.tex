\clearpage
\section{Definice ekonomie a mikroekonomie. Metody ekonomické vědy. Chyby
v ekonomickém způsobu myšlení a uvažování. Racionální chování člověka a úskalí
rozhodovacích procesů.}

\subsection{Ekonomie}
\begin{itemize}
    \item Věda o lidském rozhodování a jednání ve světě omezených zdrojů a neomezených potřeb.
    Věnuje se produkci, rozdělování a spotřebě výrobků a služeb.
\end{itemize}

\subsection{Mikroekonomie}
\begin{itemize}
    \item Analýza chování dílčích ekonomických subjektů: jednotlivců, domácností a firem,
    stav a vývoj jednotlivých trhů (výrobků a služeb, primárních výrobních faktorů).
    \item Otázky jako příklad:
    \begin{itemize}
        \item jak se utváří cena na trhu čaje
        \item jak se mění chování spotřebitele zvýšením ceny benzínu
        \item jaký vliv bude mít na pracovní úsilí lidí vyšší daň ze mzdy
        \item jak se chovají odborové svazy
        \item jak se určuje výroba v jednotlivých firmách, odvětvích apod.
        \item jak jsou určeny mzdy, zisky, úroky a jiné důchody
        \item jak se chová na trhu spotřebitel
        \item jak se projevuje státní politika na změnách výroby, ceny v jednotlivých firmách, na
        konkrétních trzích, atd.
    \end{itemize}
\end{itemize}

\subsection{Metody ekonomické vědy}
\begin{itemize}
    \item analýza (dělení celku na části)
    \item syntéza (skládání myšlenkových částí do celku)
    \item indukce (vytváření obecných pravidel na základě konkrétních příkladů)
    \item dedukce (z předpokladů vyvodíme závěr)
    \item \textbf{téměř není možné} využít metodu experimentu
    \item modely
    \item Metody nejsou univerzální ani zastupitelné, mají různé přednosti a ekonomie je velice komplexní.
\end{itemize}

\subsection{Chyby v ekonomickém způsobu myšlení}
\begin{itemize}
    \item nedodržení pravidla \uv{za jinak stejných podmínek} (ceteris paribus)
    \item omyl \uv{poté, tedy proto} (záměna příčiny a a následku)
    \item považování celku za sumu částí
    \item omyl usuzování z části na celek
    \item subjektivnost
    \item nejistota v ekonomickém životě (ekonomická pravidla platí jen v průměru, ne pro každý případ)
\end{itemize}

\subsection{Racionální chování člověka}
\begin{itemize}
    \item koncept \textbf{homo economicus} - dokonale racionální jedinec
    \item Jedná tak, aby za nejnižší možné náklady maximalizoval svůj užitek.
    \item Týká se volby způsobu dosažení subjektivních cílů, ne volby cílů.
    \item efektivita
\end{itemize}

\subsection{Úskalí rozhodovacích procesů}
\begin{itemize}
    \item Člověk dělá chyby, protože:
    \begin{itemize}
        \item opomíjí náklady obětované příležitosti (Když mám vlastní prostory, kde podnikám, přicházím o možný zisk, který bych měl, kdybych tyto prostory pronajal. To jsou obětované příležitosti.)
        \item bere v úvahu tzv. utopené náklady (Škapová Škapu ubije, protože ho nezajímá, že ona nemůže na dovolenou. Je to pro něj jen utopený náklad.)
        \item nerozlišuje rozdíl mezi průměrnými a mezními hodnotami (kolik nás bude v současné konfiguraci další stroj, může být jiné číslo než kolik je průměrný náklad na jeden stroj, příklad NASA rakety)
    \end{itemize}
\end{itemize}