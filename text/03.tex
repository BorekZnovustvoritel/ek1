\clearpage
\section{Vznik trhu. Definice trhu. Základní pravidla pro fungování tržní ekonomiky. Dělení trhů.}

\subsection{Vznik trhu}
\begin{itemize}
    \item Kvůli dělbě práce, díky němu je větší efektivita
    \item Předcházel mu barter (směna výrobků)
\end{itemize}

\subsection{Definice trhu}
\begin{itemize}
    \item Uspořádání, při kterém na sebe vzájemně působí prodávající a kupující, což vede ke stanovení cen 
    a~množství výrobků (či služeb)
    \item oblast ekonomiky, ve které dochází k směně činností mezi jednotlivými ekonomickými subjekty
    prostřednictvím směny zboží
\end{itemize}

\subsection{Základní pravidla pro fungování tržní ekonomiky}
\begin{itemize}
    \item dodržování smluv
    \item ochrana soukromého vlastnictví
    \item volný vstup na trhy
\end{itemize}

\subsection{Dělení trhů}
\begin{itemize}
    \item Podle územního hlediska:
    \begin{itemize}
        \item místní
        \item národní
        \item světový
    \end{itemize}
    \item podle počtu zboží, které na trhu sledujeme:
    \begin{itemize}
        \item dílčí (kupuje a prodává se na něm jediný druh zboží)
        \item agregátní
    \end{itemize}
    \item podle předmětu koupě a prodeje:
    \begin{itemize}
        \item trh výrobních faktorů (práce, půda, kapitál)
        \item trh peněz
        \item trh produktů (statků), pčítají se i služby
    \end{itemize}
\end{itemize}