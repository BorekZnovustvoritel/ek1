\clearpage
\section{Nedokonalá konkurence (charakteristiky, rozdíly oproti dokonalé konkurenci, příčiny
vzniku.}

\subsection{Charakteristiky}
\begin{itemize}
    \item Nejsou splněné některé z podmínek dokonalé konkurence (velké množství 
    prodávajících a nakupujících, dokonalá informovanost kupujících, homogenní produkt,
    volný vstup na trh, nulové náklady na změnu dodavatele)
    \item Alespoň jeden prodávající (firma), který může ovlivnit tržní cenu
    \item Rozdílná ekonomická síla jednotlivých výrobců v důsledku koncentrace
    \item Méně efektivní než dokonalá konkurence, protože mezní náklady firem jsou menší 
    než mezní užitek jejich produkce
\end{itemize}

\subsection{Rozdíly oproti dokonalé konkurenci}
\begin{itemize}
    \item není v \uv{tvrdosti} konkurence, ale v cenové tvorbě
    \item Firmy na nedokonale konkurenčním trhu tvoří svou cenu, kdežto firmy na dokonale
    konkurenčním trhu cenu přijímají a ceně přizpůsobují nabízené množství.
    \item Individuální poptávková křivka klesá (a v případě monopolu je totožná s tržní 
    poptávkovou křivkou) a křivka mezních příjmů (MR) klesá rychleji než křivka poptávky.
    \item Na obou trzích, dokonalém i nedokonalém platí, že firmy maximalizují zisk při takové
    ceně, při které se mezní příjem rovná mezním nákladům (MR = MC).
\end{itemize}

\subsection{Příčiny vzniku}
\begin{enumerate}
    \item rozdílná ekonomická síla jednotlivých výrobců v důsledku koncentrace
    \item odlišné nákladové a poptávkové podmínky jednotlivých odvětví
    \item vysoké bariéry vstupu (např. vysoké fixní náklady)
    \item patentová práva
    \item cla
    \item vysoké dopravní náklady
\end{enumerate}
Patrně nejdůležitějším předpokladem vzniku nedokonale konkurenčního prostředí jsou odlišné
nákladové a poptávkové podmínky jednotlivých odvětví.