\clearpage
\section{Chování výrobce – účetní a ekonomické pojetí nákladů, výnosů a zisku (explicitní
náklady, implicitní náklady, účetní zisk, ekonomický zisk, normální zisk), členění
nákladů v závislosti na velikosti objemu produkce (celkové náklad, fixní náklady,
variabilní náklady, mezní náklady, průměrné náklady, průměrné fixní náklady,
průměrné variabilní náklad) a příjmů (celkový příjem, průměrný příjem, mezní příjem).}

\subsection{Náklady}
\begin{itemize}
    \item Představují množství vstupu násobené cenou jednotky vstupu.
    \item Účetní náklady
    \begin{itemize}
        \item vynaložené náklady, jejichž pohyb je zanesen v účetních knihách
        \item jde o tzv. explicitní náklady
    \end{itemize}
    \item ekonomické náklady
    \begin{itemize}
        \item širší pojem
        \item nejen náklady explicitní, ale i náklady implicitní (náklady, 
        které firma reálně neplatí). Jejich existence je založena na principu 
        nákladů obětované příležitosti.
        \item \textbf{explicitní náklady} - platí výrobce za používání cizích výrobních 
        faktorů – platí vlastníkům těchto faktorů jejich obětované příležitosti.
        \item \textbf{implicitní náklady} - odrážejí obětované příležitosti výrobcových 
        vlastních výrobních faktorů – to co by za ně dostal v druhé nejlepší příležitosti.
    \end{itemize}
\end{itemize}

\subsection{Výnosy}
finanční prostředky získané prodejem produkce (tržby), ale i další získané finanční 
prostředky např. úroky z vkladů.

\subsection{Zisk}
\begin{itemize}
    \item Rozdíl mezi výnosy a náklady
    \item \textbf{Účetní zisk} je rozdíl mezi celkovým příjmem a explicitními náklady.
    \item \textbf{Ekonomický zisk} je rozdíl mezi celkovým příjmem a ekonomickými náklady.
    \item \textbf{Normální zisk} představuje náklady příležitosti zdrojů poskytnutých vlastníky
    firmy (normální zisk = účetní zisk – ekonomický zisk).
\end{itemize}

\subsection{Členění nákladů v závislosti na velikosti objemu produkce}
\begin{itemize}
    \item \textbf{Celkové náklady} ($TC$ – Total Cost) se skládají z variabilních nákladů a fixních nákladů.
    \item \textbf{Fixní náklady} ($FC$ - Fixed Cost), jsou takové náklady, které se se změnou
    objemu výroby nemění. Firma je musí vynakládat při každém (tedy i nulovém) objemu výroby. 
    (Odpisy budov a strojního vybavení, náklady na topení, osvětlení budov).
    \item \textbf{Variabilní náklady} ($VC$ - Variable Cost) jsou náklady, které se se změnou
    objemu výroby mění. Sem patří například přímé mzdy, náklady na přímý materiál a~energie
    bezprostředně vynaložené na zhotovení výrobků. Variabilní náklady se mohou s~objemem
    produkce měnit lineárně nebo nelineárně.
    \item \textbf{Mezní náklady} udávají přírůstek celkových nákladů vyvolaný zvýšením produkce o jednotku. Mezní náklady jsou důležité pro rozhodování o produkci firmy. $MC=\frac{\Delta TC}{\Delta Q}$
    \item \textbf{Průměrné náklady} jsou náklady na jednotku produkce: $AC=\frac{TC}{Q}$
    \item \textbf{Průměrné fixní náklady} jsou fixní náklady na jednotku produkce.
    Průměrné fixní náklady s růstem produkce klesají. $AFC=\frac{FC}{Q}$
    \item \textbf{Průměrné variabilní náklady} jsou variabilní náklady na jednotku produkce. $AVC=\frac{VC}{Q}$
\end{itemize}

\subsection{Členění příjmů v závislosti na velikosti objemu produkce}
\begin{itemize}
    \item \textbf{Celkový příjem} ($TR$ – Total Revenue) – je celková částka, 
    kterou firma získá prodejem svých výrobků.
    \item \textbf{Průměrný příjem} ($AR=\frac{TR}{Q}$ - Average Revenue) – je celková částka,
    kterou firma získá prodejem svých výrobků.
    \item \textbf{Mezní příjem} ($MR=\frac{\Delta TR}{\Delta Q}$ – Marginal Revenue) – je změna 
    celkového příjmu vyvolaná změnou vyrobeného množství o jednotku.
\end{itemize}